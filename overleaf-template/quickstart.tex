\documentclass{scrreprt}
\usepackage{everything}
\begin{document}

\section*{Quickstart}
To get started, you only need to include the package called \texttt{everything}. It will have you covered.

The package isn't built in, so you'll have to place the file \texttt{everything.sty} in the directory with your document. If you copied the Overleaf project from GitHub, the file will already be there. You're all good!

Please give the file a descriptive name. The file's name will be displayed as a clickable link on the site.

\subsection*{Sections and subsections}
The \texttt{documentclass} is a \texttt{scrreprt} and not an article, so as to mimic the display on the site. If you add a title, your file won't compile. 

To create headers, use \texttt{section*}, \texttt{subsection*} and \texttt{subsubsection*}.

\subsection*{Math commands}
All regular commands work in math mode. You could, for example, write $X: \Omega \mapsto \mathbb{R}^2$.

Double dollar expressions work too:
$$e^{i \pi} = -1$$

And should you need more lines, use the \texttt{align*} environment.
\begin{align*} 
2x - 5y &=  8 \\ 
3x + 9y &=  -12
\end{align*}

Theorems and definitions work as usual. On the website, they're displayed in boxes.

\begin{theorem}[Fundamental theorem of something]
Here goes some text, and possibly some equations.
\end{theorem}
\begin{proof}
Here goes the proof.
\end{proof}

\begin{definition}[A definition]
This is a definition.
\end{definition}

\subsection*{Misc}
You can also insert quotes. For quotes with sources, use the \texttt{somequote} command.

\somequote{Don't spread fake news.}{Plato}{The Republic}

But there's also a \texttt{simplequote} command, if you don't know the source.

\simplequote{Be kind to others.}

Happy TeXing!

\end{document}