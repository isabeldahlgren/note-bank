\documentclass{scrreprt}
\newcommand{\somequote}[3]{\begin{quotation} \textit{#1} \end{quotation} \begin{flushright} - #2, \textit{#3}\end{flushright} }
\newtheorem{theorem}{Theorem}
\begin{document}

\section{Induction}

Mathematical induction is a proof technique mainly used to prove results about natural numbers.

\subsection*{The need for induction}

Have a look at this formula.
$$\sum_{k = 0}^n k = \frac{n(n + 1)}{2}$$

How could you go about proving it?

For starters, you might verify the formula, plugging in a few values for $n$ and $k$. Then you'll get
$$n = 3, \qquad 1 + 2 + 3 = 6 = \frac{3(3 + 1)}{2}$$

How about $n = 4$?
$$n = 4, \qquad 1 + 2 + 3 + 4 = 10 = \frac{4(4 + 1)}{2}$$

Indeed, you could work through the calculations for a bunch of other $n$:s.

However, your task was to prove it for every possible $n$. This would mean performing the same calculation an infinite number of times, which clearly isn't feasible.

\subsection*{The main idea}

The following quote captures the main idea behind mathematical induction.

\somequote{Mathematical induction proves that we can climb as high as we like on a ladder, by proving that we can climb onto the bottom rung (the basis) and that from each rung we can climb up to the next one (the step).}{R. Graham, D. Knuth, O. Patashnik}{Concrete Mathematics}

\subsection*{In practise}

Inductive proofs all follow the same template.

\subsection*{Guidelines}
Here are some practical guidelines.

As a rule of thumb, whenever you're asked to prove a result about natural numbers, consider an inductive proof. Induction isn't the only proof technique out there, but it might be your best bet.



\end{document}