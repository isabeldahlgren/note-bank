\documentclass{scrreprt}
\usepackage{everything}
\begin{document}

\section{Inverse transform sampling}

How do you generate a random sample? Hey, importing a library counts as cheating. Enter inverse transform sampling.

\begin{theorem}[Inverse transform sampling]
    To generate a random sample $x_1, \dots, x_n$, generate $u_1, \dots, u_n$ from $\text{U}(0, 1)$ and let $x_i = F_X^{-1}(u_i)$ for $i = 1, \dots, n$.
\end{theorem}

There are two ways to understand this result.

\subsection*{The CDF proof}
\begin{proof}
    If we can show that the random variable $Y = F_X^{-1}(U)$ has the same CDF as $X$, we're done. And as it turns out,
    $$F_Y(y) = P(F_X^{-1}(U) < y) = P(U < F_X(y)) = F_X(y)$$
    Here, we've used the fact that the CDF is monotonic and increasing.
\end{proof}

\subsection*{The transformation proof}
\begin{proof}
    In this proof, we shall transform the random variable $Y = F_X^{-1}(U)$ and show that its PDF is $f_x(y)$.
    $$f_Y(y) = f_U(F_X(y)) \cdot \dfrac{dF_X}{dy} = f_X(y)$$
\end{proof}

\end{document}