\documentclass{scrreprt}
\usepackage{everything}
\begin{document}

\section*{Polynomials}

\subsection*{Polynomial division}

The booklet discusses two ways to perform polynomial division.

\begin{enumerate}
    \item Polynomial long division (sv. liggande stolen)
    \item The Ansatz Method (quick disclaimer: this is a made-up name)
\end{enumerate}

Here's how the Ansatz Method works. If you want to divide $f(x)$ by $g(x)$, introduce a third polynomial for the quota, $q(x)$. Then expand $g(x) \cdot q(x)$ and do some pattern-matching to find the coefficients of $q(x)$. See example 3.8 for details.

Polynomial long division is shown in example 3.3, where the polynomial has real coefficients. Then the Ansatz Method is used in example 3.8, where the polynomial has complex coefficients. However, both methods work for both kinds of polynomials.

\simplequote{Bottom line: use whichever method you think is the easiest.}

\end{document}