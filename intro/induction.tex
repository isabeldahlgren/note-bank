\documentclass{scrreprt}
\usepackage{everything}
\begin{document}

\section*{Induction}
Mathematical induction is a proof technique used to prove patterns.

\subsection*{The need for induction}

Have a look at the sum formula:
$$\sum_{k = 1}^n k = \frac{n(n + 1)}{2}$$
How could you prove it for every $n \geq 1$?

Of course, you might verify the formula, plugging in a few values for $n$. For example,
$$n = 3: \qquad 1 + 2 + 3 = 6 = \frac{3(3 + 1)}{2}$$
However, your task was to prove it for every possible $n$. You'd have to perform the same calculation an infinite number of times.

Induction uses a bit of logic to sidestep this problem.

Firstly, you should check that the formula holds for $n = 1$.
$$\sum_{k = 1}^n k = 1 = \frac{1(1 + 1)}{2}$$
Secondly, assume that the formula holds for some particular $n$. For $n + 1$, you get
\begin{align*}
    \sum_{k = 1}^{n + 1} k &= \underbrace{1 + ... + n}_{n(n + 1)/2 \text{, by assumption}} + (n + 1) \\
    &= \frac{n(n + 1)}{2} + (n + 1) = \frac{n(n + 1) + 2 \cdot (n + 1)}{2}\\
    &= \frac{(n + 1)(n + 2)}{2} = \frac{(n + 1)((n + 1) + 1)}{2} \\
\end{align*}
But the final expression is just the sum formula for $n + 1$. Hence, if the formula is true for $n$, it is also true for $n + 1$. This finding, along with the fact that the formula is true for $n = 1$, actually guarantees that the formula is true for all $n \geq 1$.

This might seem like a leap of faith, but it can be proved with predicate logic.

Induction can also be likened to a domino effect. \simplequote{If the first domino brick falls, and a domino brick is heavy enough to tip the next, every domino brick will fall.} In this example, you began by proving that the first domino brick could be tipped. Then you showed that a domino brick could cause the next to topple over.

\subsection*{The Principle of Induction}
Inductive proofs follow the same template.

\begin{theorem}[The Principle of Induction]
    Let $n$ be an integer, and let $P(n)$ be a statement about $n$. The principle of induction is a way of proving that $P(n)$ is true for all integers $n \geq n_0$, where $n_0$ is some integer. It involves two steps:
    \begin{enumerate}
        \item Base case: show that $P(n_0)$.
        \item Inductive step: show the implication $P(n) \Rightarrow P(n + 1)$.
    \end{enumerate}
    Then you can conclude that $P(n)$ for all $n \geq n_0$.
\end{theorem}
The assumption that $P(n)$ from the inductive step is called the inductive hypothesis.

As a rule of thumb, whenever you're dealing with natural numbers, consider an inductive proof.

\subsection*{Proving divisibility}
In this course, you will use induction to prove formulas, inequalities and divisibility. Formulas and inequalities are pretty straightforward.

But how would you prove that $6$ divides $8^n - 2^n$ for every $n \in \mathbb{N}^+$?

To begin with, specify what you're asked to prove. If $6$ divides $8^n - 2^n$, you can pull out a factor $6$ from $8^n - 2^n$. Specifically,
\begin{definition}
    Let $a, b \in \mathbb{Z}$. $a$ is divisible by $b$ if there exists a $c \in \mathbb{Z}$ such that $a = b \cdot c$.
\end{definition}
In this definition, $c$ is just the quota. The definition says that there must exist a quota, so we can write $a = b \cdot c$.

Definitions are important because otherwise you don't know what to prove. Also, definitions might hint how you could proceed. In this case, your goal is to show that $8^n - 2^n = 6 \cdot c$.

\begin{enumerate}
    \item Base case: $8^1 - 2^1 = 6 = 6 \cdot \underbrace{1}_{c}$.
    \item Inductive step: suppose that $8^n - 2^n = 6c$, for some $c$. Then

    \begin{align*}
        & 8^{n + 1} - 2^{n + 1} \\
        =& 8^n \cdot 8 - 2^n \cdot 2 \\
        =& \underbrace{8^n - 2^n}_{6c} + \underbrace{8^n - 2^n}_{6c} + 8^n \cdot 6 \\
        =& 6 \cdot \underbrace{(c + c + 6)}_{c'} \\
        =& 6 \cdot c'
    \end{align*}
    
\end{enumerate}
That's it. In the definition, there's nothing special about the letter $c$. If you can prove that there exists a number, call it $c$, $d$ or whatever you may, such that $a = b \cdot (\text{the number})$, you're good.

\subsection*{Additional material}
Induction can be applied to other kinds of problems too. For example, suppose $f(x) = \frac{1}{x}$. Then you could use induction to prove that $$f^n(x) = (-1)^n \cdot n! \cdot x^{-(n + 1)}$$ Can you see how?

\end{document}