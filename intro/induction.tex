\documentclass{scrreprt}
\usepackage{everything}
\begin{document}

\section*{Induction}
Mathematical induction is a proof technique used to prove patterns.

\subsection*{The need for induction}
Have a look at the following formula.
$$\sum_{k = 1}^n k = \frac{n(n + 1)}{2}$$
How could you prove it for every $n \geq 1$?

For starters, you might verify the formula, plugging in a few values for $n$. For example,
$$n = 3: \qquad 1 + 2 + 3 = 6 = \frac{3(3 + 1)}{2}$$
You could work through the calculations for a bunch of other $n$:s. However, your task was to prove it for every possible $n$. You would have to perform the same calculation an infinite number of times.

\subsection*{Inductive proof of the sum formula}
Induction uses a bit of logic to sidestep this problem.

Firstly, you should check that the formula holds for $n = 1$.
$$\sum_{k = 1}^n k = 1 = \frac{1(1 + 1)}{2}$$
Secondly, assume that the formula holds for some particular $n$. For $n + 1$, you get
\begin{align*}
    \sum_{k = 1}^{n + 1} k &= \underbrace{1 + ... + n}_{n(n + 1)/2 \text{, by assumption}} + (n + 1) \\
    &= \frac{n(n + 1)}{2} + (n + 1) = \frac{n(n + 1) + 2 \cdot (n + 1)}{2}\\
    &= \frac{(n + 1)(n + 2)}{2} = \frac{(n + 1)((n + 1) + 1)}{2} \\
\end{align*}
But the final expression is just the sum formula for $n + 1$. Hence, if the formula is true for $n$, it is also true for $n + 1$. This finding, along with the fact that the formula is true for $n = 1$, actually guarantees that the formula is true for all $n \geq 1$.

This might seem like a leap of faith, but it can be proved with predicate logic.

Induction can also be likened to a domino effect. \simplequote{If the first domino brick falls, and a domino brick is heavy enough to tip the next, every domino brick will fall.} In the example, you began by proving that the first domino brick could be tipped. Then you showed that a domino brick could cause the next to topple over.

\subsection*{The Principle of Induction}
Inductive proofs follow the same template.

\begin{theorem}[The Principle of Induction]
    Let $n$ be an integer, and let $P(n)$ be a statement about $n$. The principle of induction is a way of proving that $P(n)$ is true for all integers $n \geq n_0$, where $n_0$ is some integer. It involves two steps:
    \begin{enumerate}
        \item Base case: show that $P(n_0)$.
        \item Inductive step: assume that $P(n)$, and show that this leads to $P(n + 1)$.
    \end{enumerate}
    Then you can conclude that $P(n)$ for all $n \geq n_0$.
\end{theorem}

There are may proof techniques out there, but induction may prove useful if you're dealing with natural numbers.

\subsection*{Additional material}
In this course, you will use induction to prove formulas, inequalities and divisibility.

But induction can be applied to other kinds of problems too. For example, suppose $f(x) = \frac{1}{x}$. Then you could use induction to prove that $$f^n(x) = (-1)^n \cdot n! \cdot x^{-(n + 1)}$$ Can you see how?

\end{document}